%\section*{Teaching Philosophy}
%%%%%%%%%%%%%%%%
%My primary objective as an educator is to impart that biology is an open-ended, inquiry-based endeavor that inherently requires critical thinking and creativity.
As an undergraduate tutor, graduate student teaching assistant, workshop
instructor, and lecturer,
I have had the privilege of teaching a variety of biology subjects to
undergraduate and graduate students, postdocs, and professors.
While my approach to each subject varies, my primary philosophy remains the
same:
\textbf{\textit{My objective as an educator is to impart that biology is an
open-ended, inquiry-based endeavor that inherently requires critical thinking
and creativity.}}
As with all fields of science, it is not a subject that can be effectively
learned from a textbook alone.
I try to blur the distinction between teaching and research and incorporate
active, inquiry-based learning techniques and primary literature into my
teaching at all levels of undergraduate and graduate education.
My goal is not only to teach the subject material, but also expose students to
the venture of science, and hopefully inspire students to consider a
research-based career. 

\section*{My approach}
My general approach to teaching can be summarized in five main components:
(1) preparation,
(2) establishing a interactive learning environment,
(3) conveying concepts,
(4) making the material relevant,
and
(5) measuring success.

\subsubsection*{Preparation}
% The foundation of good teaching is knowledge and preparation.
I always strive to understand the concepts behind, and the literature beyond,
the course material.
Without a thorough understanding of the subject, good teaching is impossible.

\subsubsection*{Establish learning environment}
Establishing a comfortable, interactive learning environment in the classroom
takes practice and is different for every group of students.
I accomplish this with enthusiasm in my presentation of the material, and
constantly breaking up lectures with questions and active-learning activites.
% One effective strategy I have found is to be enthusiastic and engaging in my
% presentation of the material; this helps make the experience more enjoyable for
% the students and myself, and is enough to generate classroom participation in
% most cases.
% Using some appropriate humor can aid this process; I have found that making fun
% of myself in a lighthearted manner can be very effective for creating a
% comfortable environment for participation and help make me approachable to the
% students outside of class.
% When I get a group of students that are reluctant to participate, I kindly
% force participation and reward it with positive feedback.
% I do this by frequently asking questions and waiting for their feedback.
% Once the students realize their feedback is required to proceed and is never
% received negatively, even the shyest group of students will get involved.

\subsubsection*{Convey concepts}
My goal in the classroom, review sessions, and during one-on-one consultations
is to convey the important principles and concepts behind the material.
I will do everything possible to prevent students from simply memorizing
information from course materials.
% I have two main strategies for helping students truly understand the material.
To accommodate the diversity of ways in which students learn effectively, I
often present concepts from several different perspectives.
In order to encourage creativity and synthesis of newly acquired knowledge,
I constantly ask questions and incorporate inquiry-based activities that
require the students to go beyond the class material and build upon the
concepts they are learning.
% These questions usually have many correct answers, and so at the same time,
% this strategy can help reinforce participation, because every contribution has
% value.
% The key is to wait for the students to work it out themselves; I will often
% rephrase the question or give hints to guide their path of thought, but I never
% answer the question for them.
% To augment these questions, I incorporate inquiry-based activities and
% information from the primary literature as often as possible. 

\subsubsection*{Make material relevant}
At every opportunity, I relate the material we are discussing to current
events, my research, or the research of other members of the department.
This provides an opportunity for the students to get to know more about what I
and other people in the department do, and illustrates why the material is
important and relevant to everyday life.
I also include current literature when possible, to convey to the students that
topics in biology are open-ended and constantly advancing.
% Most importantly, I incorporate hands-on activities to teach topics, because
% there is no better way to make a subject relevant than personal experience.

\subsubsection*{Measuring learning success}
As a scientist, I think it is imperative to take an evidence-based approach to
education.
I use conceptual inventories to collect quantitative data on how well students
are learning the core concepts.
This allows me to continually improve my content, presentation, and activities
in a rigorous manner, and to test for general patterns of teaching
effectiveness.

\section*{Teaching Experience}
%%%%%%%%%%%%%%%%%
\subsubsection*{In the classroom}
As an undergraduate tutor, I helped students with special needs understand
key concepts in biology and chemistry classes.
By working with a diversity of students in one-on-one consultations, I learned
that every student has a unique way of effectively learning.
Thus, early on in my career, I realized the need to present ideas from a
variety of perspectives to accommodate diverse of learning styles.

As a graduate student, I have taught Herpetology, Biostatistics, and Genetics
to upper-level undergraduate students, and also presented guest lectures for a
graduate-level Evolutionary Biology course.
These courses have exposed me to a diversity of teaching formats, including
formal lectures, discussions, and field and computer-based labs.
I have consistently received positive teaching evaluations from students, and
was awarded the Kenneth B. Armitage Award for Excellence in Teaching for
teaching Genetics; an award given out each year to an outstanding graduate
teaching assistant (TA) in the University of Kansas Division of Biological
Sciences that is nominated by their students.

% As a teaching assistant (TA) for Herpetology, I was responsible for developing,
% organizing, and presenting the weekly lab section of the course.
% This involved preparing handouts and lab exams, presenting introductory
% lectures on the biology of different taxonomic groups, teaching specimen-based
% identification at the family and species level using taxonomic keys, and
% leading both on and off-campus field trips to collect and identify reptiles and
% amphibians.
% I had a tremendous amount of fun teaching this class, and greatly look forward
% to developing my own course in Herpetology.

% As a TA for Genetics, I was in charge of three weekly discussion sections each
% semester.
% During discussions, I would give short lectures on topics not covered in the
% main lecture section, and help the students through a series of problem-solving
% exercises that required them to synthesize the knowledge they were acquiring
% from lecture.
% As a TA for Biostatistics, I was in charge of leading inquiry-based computer
% labs.
% Each week, we would read a short summary about preliminary research (sometimes
% based on the literature and other times fictional), and I would help the
% students formulate a null to explain the observations in the reading.
% We would then either use data from the literature, or collect data using props
% (e.g., decks of playing cards) or computer simulations, and use a statistical
% technique the students had learned about in lecture the previous week to test
% the null hypothesis.
% I would then help the students interpret the results and write a formal
% statement of their findings.
% I often split the students up into groups, to help them understand the
% important concepts of estimation error and sampling distributions.

% Statistics and genetics are inherently difficult subjects for most
% undergraduate students.
% From my experiences in teaching these courses, I realized the importance of
% preparation and clearly articulating complex concepts in a variety of different
% ways.
% I also learned that establishing an interactive environment in the classroom is
% essential to discern whether students are understanding the material.
% I am excited about developing and teaching my own upper-level undergraduate
% courses, because I like the challenge of effectively teaching difficult
% concepts to a diversity of students.

I have led and assisted in the instruction for several workshops with the
common goal of teaching graduate students, postdocs, and faculty basic
principles of computer programming and bioinformatics.
I gained experience developing inquiry-based exercises and tutorials
to help students actively learn difficult concepts.
I really enjoyed breaking away from my role as a lecturer to become an active
participant in the exercises along with the students.

As a Postdoctoral Fellow at the University of Washington (UW), I have
volunteered to lecture for Applied Phylogenetics (BIOL 449) and Biogeography
(BIOL 470).
Furthermore, for Applied Phylogenetics, I developed and led an inquiry-based
computer lab on phylogenetic inference, and developed a learning-assessment
survey for the course.
The goal of the survey is to establish baseline data on how well students are
understanding the fundamental concepts of phylogenetics, and, over time, allow
for assessing the effectiveness of different teaching strategies and exercises.

I am currently lecturing
\href{http://courses.biology.washington.edu/biol180/}{Introductory Biology}
(BIOL 180) at the UW under the mentorship of Dr.\ Scott Freeman.
% BIOL 180 is the largest course in UW history, with more than 1200 students
% enrolled.
With Dr.\ Freeman and the UW \href{http://uw-berg.wikifoundry.com/}{Biology
    Education Research Group}, I am developing and implementing active-learning
educational activities aimed at improving student learning of important
concepts in biology.
I am also using conceptual inventories to quantify the effectiveness of these
activities.
All of this work is evidence-based, guided by the long history of research on
biology education at the UW.

\subsubsection*{Mentoring}
As an undergraduate student, I got involved in research projects under the
mentorship of three different biology faculty members.
These opportunities provided me with indispensable research experience as an
aspiring young scientist.
Understanding how critical my undergraduate research experience was in
preparing me for graduate school, I am always excited to get motivated
undergraduates involved in my research.

As a Ph.D. student at the University of Kansas, I have had the privilege to
mentor local high school students and KU undergraduates.
I trained high school students and undergraduates in the methods of field
collection and specimen preparation.
These students volunteered in an NSF EPSCoR project to help collect specimens,
genetic samples, and ecological data from reptiles and amphibians from across
the Central Great Plains.
Two of the high school students subsequently enrolled in KU, and under my
supervision, were employed as curatorial assistants in the KU Herpetology
Collection.
I mentored these students in the methods of collections management, including
specimen preservation, accessions and cataloging, relational database
management, genetic resources management, specimen and tissue loans, and
collections upkeep.
Furthermore, I helped Liz Lusher obtain a KU Undergraduate Research Award to
study the phylogeography of ringneck snakes (\emph{Diadophis punctatus}) across
Kansas.
With my guidance, she successfully completed the project and presented her
results at the KU Undergraduate Research Symposium.

During my fieldwork in Southeast Asia, I have mentored students in the methods
of field-based biodiversity research, including undergraduate and graduate
students from Universiti Kebangsaan Malaysia, Universiti Sains Malaysia, and La
Sierra University.
I found this experience to be particularly rewarding, and I aspire to get
undergraduate and graduate students involved with international and local
fieldwork projects as often as possible.

\section*{Prospective Courses}
%%%%%%%%%%%%%%%%%

I am excited to develop and teach my own courses, because they give me the
opportunity to inspire young students to pursue science.
I owe my career in biology to my introductory biology professor, Scott Snyder,
whose enthusiastic teaching style convinced me as a freshman undergraduate to
pursue a biology major and undergraduate research projects.
I aspire to provide the same positive experience for young students.

My research and teaching experiences have prepared me to teach a number of
courses for undergraduate and graduate students.
I am comfortable teaching a variety undergraduate classes, including
Introductory Biology, Zoology, Herpetology, Genetics, Evolution, and
Biostatistics.
For more advanced undergraduate and graduate students, I would like to develop
courses that focus on theory and methods of statistical genetics,
bioinformatics, biologically focused computer programming, and advanced
statistical topics like maximum likelihood and Bayesian statistics.

%\section*{Teaching Appointments \& Evaluation Comments}
%%During my time at KU, I have had the privilege of assisting in the teaching of introductory courses in Biostatistics (Biology 570) and Genetics (Biology 350).
%%In the Fall of 2009, I was responsible for three lab sections of Biostatistics.
%%In Spring and Fall of 2010, I was responsible for two and three discussion sections of Genetics.
%%Below are a selection of comments from my students from each appointment.
%%\begin{multicols}{2}
%\subsection*{Biology 350---Genetics---Fall 2010}
%\begin{myItemize}
%\sffamily
%\item ``Thank you so so much for reviews\ldots If you want to be in academics later, you'll be an excellent professor.''
%\item ``He was an awesome TA! Loved his review sessions---very helpful''
%\item ``The instructor presented information \underline{very} well, answered questions clearly and took the time to give thorough reviews for exams.''
%%\item ``The instructor always took time to answer questions about homework and went out of his way to help students study for exams in the class.''
%\item ``Jamie was open to questions \& explained things in a way that was easy to understand.''
%%\item ``Made hard concepts easy to understand.''
%\item ``Very helpful; often scheduled review sessions on his own time.''
%\item ``Very approachable---had lots of knowledge of material \& explained topics very well.''
%%\item ``Jamie was enthusiastic \& helpful.''
%\item ``Really great teacher. Explained things well. Better than the professors.''
%%\item ``Instructor explained well, made sure class understood material. Enjoyed the instructor, class was great!''
%%\item ``His teaching is good. He is easy to understand.''
%%\item ``He was very helpful and made himself very available for questions/reviews.''
%\end{myItemize}
%
%\subsection*{Biology 350---Genetics---Spring 2010}
%\begin{myItemize}
%\sffamily
%\item ``Jamie is the best TA!''
%\item ``Jamie is one of the best TAs I've had. Very
%helpful \& explains things very well. He never just assumed or
%expected that we knew things\ldots he took the time to explain it.''
%\item ``Jamie is one of the best TAs I've ever had. He explains the subject
%very well, answers questions and always ready to help.''
%%\item ``Very clear in his teaching. Very knowledgeable about the material.''
%\item ``Would try several ways to explain topics that were not easily
%understood.''
%%\item ``Always helpful, appreciated that he went over homework, and explained things thoroughly.''
%%\item ``He explained concepts in simple terms \& was able to clarify questions from lecture.''
%\item ``Very real \& down to earth. Available \& explains concepts well.''
%\item ``I like how Jamie is organized in discussion and also that he explains the material very well.''
%\item ``Course is well organized, interactive. Best instructor ever!''
%%\item ``I really enjoyed the class, and I appreciated the help during office hours. Also, very understanding about class conflicts.''
%\item ``He did a good job teaching a class during a difficult time---late in the day when students don't have a long attention span.''
%%\item ``Jamie is very helpful \& always responds to students needs promptly.''
%%\item ``Very calm, nice \& very helpful.''
%\item ``Jamie is a great TA. Easy to understand, knows his stuff, willing to
%take time and explain topics.''
%%\item ``He was responsive to emails \& explained things well.''
%\end{myItemize}
%
%\subsection*{Biology 570---Biostatistics---Fall 2009}
%\begin{myItemize}
%%\begin{multicols}{2}
%\sffamily
%\item ``Jamie was really nice and patient with everyone.''
%%\item ``Attentive, helpful.''
%%\item ``Taught very well.''
%\item ``Easy to talk to, approachable.''
%\item ``Very helpful in lab.''
%\item ``He was very helpful, helped me learn the material, and seemed to care
%about everyone learning and doing well.''
%%\item ``Instructor knew the material well.''
%%\item ``Willing to meet outside class.''
%\item ``Jamie was really patient and explained things very clearly.''
%%\item ``Helpful and knowledgeable TA.''
%\item ``Always willing to go out of his way to help people and make sure
%everyone is on the same page.''
%\item ``Jamie was an excellent TA. Did a great job of explaining.''
%%\item ``Lots of office hours making it easy to get help.''
%%\item ``He was very helpful and clear.''
%%\end{multicols}
%\end{myItemize}
%%\end{multicols}

