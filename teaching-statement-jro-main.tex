%\section*{Teaching Philosophy}
%%%%%%%%%%%%%%%%
%My primary objective as an educator is to impart that biology is an open-ended, inquiry-based endeavor that inherently requires critical thinking and creativity.
As a tutor, teaching assistant, workshop
instructor, professor, and prison-education volunteer,
I have had the privilege of teaching a variety of biology subjects to a
diversity of learners.
While my approach to each subject and group of students varies, my primary
philosophy remains the same:
\textbf{\textit{My objective as an educator is to impart that
        % biology
        science
        is an open-ended, inquiry-based endeavor that inherently requires
        critical thinking and creativity.}}
% As with all fields of science, it is not a subject that can be effectively
% learned from a textbook alone.
I try to blur the distinction between teaching and research and incorporate
active, inquiry-based learning techniques and primary literature into my
classes at all levels of
% undergraduate and graduate
education.
My goal is not only to teach the subject material, but also expose students to
the venture of science, and hopefully inspire students to consider a
career in research.

% \subsection*{My Approach}
My general approach to teaching can be summarized in five components:
% (1) preparation,
% (2) establishing an interactive learning environment,
% (3) conveying concepts,
% (4) making the material relevant,
% and
% (5) measuring success.
\textbf{(1) Preparation}---I always strive to understand the concepts behind,
and the literature beyond, the course material.
% Without a thorough understanding of the subject, good teaching is impossible.
\textbf{(2) Establish learning environment}---I use enthusiasm, discussion
questions, and active-learning exercises to establish a comfortable,
interactive learning environment in the classroom.
% One effective strategy I have found is to be enthusiastic and engaging in my
% presentation of the material; this helps make the experience more enjoyable for
% the students and myself, and is enough to generate classroom participation in
% most cases.
% Using some appropriate humor can aid this process; I have found that making fun
% of myself in a lighthearted manner can be very effective for creating a
% comfortable environment for participation and help make me approachable to the
% students outside of class.
% When I get a group of students that are reluctant to participate, I kindly
% force participation and reward it with positive feedback.
% I do this by frequently asking questions and waiting for their feedback.
% Once the students realize their feedback is required to proceed and is never
% received negatively, even the shyest group of students will get involved.
\textbf{(3) Convey concepts}---My goal is to convey the important principles
and concepts behind the material, and prevent students from simply memorizing
information.
% I have two main strategies for helping students truly understand the material.
To accommodate the diversity of ways in which students learn effectively, I
often present concepts from several different perspectives.
In order to encourage creativity and synthesis of newly acquired knowledge,
I constantly ask questions and incorporate inquiry-based activities that
require the students to go beyond the class material and build upon the
concepts they are learning.
% These questions usually have many correct answers, and so at the same time,
% this strategy can help reinforce participation, because every contribution has
% value.
% The key is to wait for the students to work it out themselves; I will often
% rephrase the question or give hints to guide their path of thought, but I never
% answer the question for them.
% To augment these questions, I incorporate inquiry-based activities and
% information from the primary literature as often as possible. 
\textbf{(4) Make material relevant}---At every opportunity, I relate concepts
to current events, my research, or the research of other members of the
department.
This provides an opportunity for the students to learn about ongoing
research at their institution and how the material is important and relevant.
I also include current literature to convey to the students that topics in
biology are open-ended and constantly advancing.
% Most importantly, I incorporate hands-on activities to teach topics, because
% there is no better way to make a subject relevant than personal experience.
\textbf{(5) Measuring learning success}---I use conceptual inventories to
collect quantitative data on how well students are learning core concepts.
This allows me to test for general patterns of teaching effectiveness, and take
an evidence-based approach to my content, presentation, and exercises.

\section*{Teaching experiences at the UW}
As an NSF postdoctoral research fellow at the University of Washington,
I was able to secure an extra year of funding to teach
\href{http://courses.biology.washington.edu/biol180/}{Introductory Biology}
(BIOL 180) in Fall 2014 and Spring 2015;
the former was the largest course in UW history, with more than 1200 students
in two sections.
% BIOL 180 is the largest course in UW history, with more than 1200 students
% enrolled.
Under the mentorship of Dr.\ Scott Freeman and the UW
\href{https://sites.google.com/site/uwbioedresgroup/home}{Biology Education
    Research Group}, I developed and implemented active-learning
educational activities aimed at improving student learning of important
concepts in biology,
and used conceptual inventories to quantify the effectiveness of these
activities.
All of this work was evidence-based, guided by the long history of research on
biology education at UW.

\subsubsection*{UW teaching evaluations}
UW student evaluations report an overall index of the quality of the class that
represents the ``combined responses of students to global summative items.''
They also include the Challenge and Engagement Index (CEI), which combines
student responses to items relating to how challenging students found the
course and how engaged they were.
My scores for these metrics are shown below, with the number of reporting
students in parentheses;
for both metrics, a higher number is better.
\begin{center}
\begin{tabular}{l l c c}
    \hline
    \textbf{UW Course} & \textbf{Semester} & \textbf{Class quality index} & \textbf{CEI} \\
                       &                   & Scale: 0--5                  & Scale: 1--7 \\
    \hline
    BIOL 180 A: Intro Bio & Fall 2014 & 4.5 (530) & 5.6 (530) \\
    BIOL 180 B: Intro Bio & Fall 2014 & 4.6 (443) & 5.7 (530) \\
    BIOL 180 A: Intro Bio & Spring 2015 & 4.0 (145) & 5.2 (145) \\
    \hline
\end{tabular}
\end{center}

\section*{Teaching experiences at Auburn University (AU)}
To improve graduate student education at AU,
I am a member of the Informatics Steering Committee that developed a
curriculum for our students to earn a
\href{http://bulletin.auburn.edu/thegraduateschool/graduatedegreesoffered/biologicalsciencesmsphd_major/computationalbiology_gradcert/}{graduate certificate in computational biology}.
For my department, I have helped establish student learning outcomes for our
majors, along with assessment tools for these outcomes.
For our annual
\href{http://www.auburn.edu/cosam/bioinformatics/}{Bioinformatics Bootcamp},
I developed several active-learning exercises,
including a module on using version control to improve reproducibility in
science that involves a
\href{http://phyletica.org/slides/git-intro/}{``hands-on'' lecture}
followed by a
\href{https://github.com/joaks1/au-bootcamp-git-intro}{group exercise}.
Due to these efforts, I was invited to participate in a symposium at the
iEvoBio 2019 Conference about how best to use computation for teaching biology,
which we have synthesized into a
white paper\footnote{\label{ievobiopaper}\shortfullcite{Wright2019}}.

During my time at AU,
I have developed and taught three graduate-level courses
% evolution,
% scripting for biologists and
% advanced statistical methods for evolutionary genetics,
and two predominantly undergraduate courses,
% evolution and
% herpetology.
which are summarized below.
For my mid-tenure review earlier this year, I was ranked as ``exceeding
expectations'' for teaching these courses.

\subsubsection*{Scripting for Biologists (BIOL 7180)}
An in-depth introduction for graduate students to the fundamentals of scripting
and other computational skills required to make biological insights from ``Big
Data.''
My approach to this class involves both
\href{http://phyletica.org/slides/python/intro/}{instructor-led active-coding
    sessions}
and
\href{https://github.com/joaks1/python-translation-project}{goal-oriented group
    exercises}.
% Co-developed and co-taught with Dr.\ Scott Santos.

\subsubsection*{Advanced statistical methods for evolutionary genetics (BIOL 7960)}
A literature-based class for graduate students to become familiar with the
statistical underpinnings of model-based comparative methods for testing a
broad range of evolutionary hypotheses, from molecular evolution to community
assembly.
% Co-developed and co-taught with Dr.\ Tonia Schwartz.

\subsubsection*{Herpetology (BIOL 5740/6740)}
A course for undergraduate and graduate students that focuses on the ecology,
physiology, evolution, diversity, and systematics of reptiles and amphibians.
I developed this course to focus on fundamental concepts of ecology and
evolution, using reptiles and amphibians as a model.
% Co-developed and co-taught with Dr.\ Dan Warner.

\subsubsection*{Evolution \& Systematics (BIOL 3030)}
A core course for our undergraduate students (mostly sophomores and juniors) to
explore the theory that unifies the form and function of all life.

\subsubsection*{Evolution (BIOL 7200)}
A graduate-level evolution course developed in collaboration with other biology
faculty.
The goal of the course is to help ensure that our incoming cohorts of graduate
students have a solid understanding of the fundamental principles of
evolutionary biology.

\subsubsection*{AU teaching evaluations}
AU evaluations present students with a variety of statements about the
course,
which they rank from zero (low) to six (high).
Below, I summarize my scores for two of these prompts:
(1) ``The instructor's overall performance was \ldots''
and
(2) ``My overall learning in the class was \ldots''
\begin{center}
\begin{tabular}{l l c c}
    \hline
    \textbf{AU Course} & \textbf{Semester} & \textbf{Overall effectiveness} & \textbf{Overall learning} \\
                       &                   & Scale: 0--6                    & Scale: 0--6 \\
    \hline
    BIOL 3030: Evolution \& Systematics & Fall 2019 & 5.8 (20) & 5.4 (20) \\
    BIOL 3030: Evolution \& Systematics & Fall 2018 & 5.3 (27) & 5.3 (27) \\
    BIOL 3030: Evolution \& Systematics & Fall 2016 & 4.7 (18) & 4.7 (18) \\
    BIOL 5740/6740: Herpetology & Spring 2020 & 5.8 (20) & 5.6 (20) \\
    BIOL 5740/6740: Herpetology & Spring 2018 & 5.8 (7) & 5.9 (7) \\
    BIOL 5740/6740: Herpetology & Spring 2017 & 5.7 (10) & 5.7 (10) \\
    BIOL 7180: Scripting for Biologists & Spring 2019 & 4.8 (4) & 4.8 (4) \\
    \hline
\end{tabular}
\end{center}

\section*{Prison Education}
To bring science education to more diverse and underserved learners,
% than I see in my courses on the AU campus,
I have been working with
the Alabama Prison + Arts Education Project (APAEP) to develop and teach three
14-week courses in evolutionary biology to adult prisoners in correctional
facilities across Alabama.
The classes provided by APAEP have predominantly focused on the
arts and humanities, and
my
% long-term
goal is to incorporate a broad set of STEM courses into the
curriculum.
I am currently seeking funding in the outreach components of my
NSF proposals
to incorporate more computationally focused classes.
My students in Alabama correctional facilities are exceptionally hard-working
and eager to learn.
Working with them has been a privilege and, by far, my most rewarding
experience as a teacher.

\section*{Mentoring}
Getting to work with a diverse group of talented undergraduate and graduate
students and postdocs in my lab is the most rewarding part of my job as a
PI.
I work with each of my lab members to develop an individual development
plan\footnote{\label{idp}\shortfullcite{myIDPonline}} that fits
their level of experience, learning style, and career goals.
Most importantly, I try to create an atmosphere in my lab where
lab members work \emph{with} me, as opposed to \emph{for} me.

I am incredibly lucky to have recruited such an amazing group of students and
postdocs, who make
% my job easy and
our lab group fun and productive.
While at AU, I have graduated two M.Sc.\ students, Aundrea Westfall
and Breanna Sipley.
Aundrea is now a Ph.D.\ student at the University of Texas Arlington,
and
Breanna is a Ph.D.\ student and NSF Graduate Research Fellow at the University
of Idaho.
I currently have seven Ph.D. students in my lab, two in their 5th
year (Kerry Cobb and Randy Klabacka), four in their 2nd year
(Tashitso Anamza, Matt Buehler, Tanner Myers, and Claire Tracy),
and one in her 1st year (Morgan Muell).
I also serve on the advisory committee of 10 other graduate students.
We have had 14 undergraduate student researchers in our lab, including four
NSF-funded REU students.
Our first cohort of undergraduate researchers, Charlotte Benedict, Ryan Cook,
and Miles Horne, are now in biology graduate programs at the Ohio State
University, Villanova University, and the University of Texas at El Paso.
% Three of these students are still working in the lab,
% and two have moved on to graduate programs.
I have mentored three postdocs, Jesse Grismer, Brian Folt, and Perry Wood, Jr.
Jesse is now an Assistant Professor at La Sierra University,
Brian is in a second postdoc position,
and Perry is still working in my lab, funded by an NSF award
(\href{https://www.nsf.gov/awardsearch/showAward?AWD_ID=1656004&HistoricalAwards=false}{DEB 1656004}).


\section*{Prospective Courses}
%%%%%%%%%%%%%%%%%

% I am excited to develop and teach my own courses, because they give me the
% opportunity to inspire young students to pursue science.
% I owe my career in biology to my introductory biology professor, Dr.\ Scott
% Snyder, whose enthusiastic teaching style convinced me as a freshman
% undergraduate to pursue a biology major and undergraduate research projects.
% I aspire to provide the same positive experience for young students.

My research and teaching experiences have prepared me to teach a number of
courses for undergraduate and graduate students.
I am comfortable teaching a variety of undergraduate classes, including
biostatistics,
biogeography,
evolution,
genetics,
herpetology,
% zoology,
and
introductory biology.
For more advanced undergraduate and graduate students, I enjoy teaching courses
that focus on theoretical and computational aspects of
phylogenetics,
% evolutionary genetics,
evolutionary genomics,
% molecular ecology,
bioinformatics,
% biologically focused computer programming,
and
advanced statistical topics like
maximum likelihood,
Bayesian inference,
and
machine learning.

%\section*{Teaching Appointments \& Evaluation Comments}
%%During my time at KU, I have had the privilege of assisting in the teaching of introductory courses in Biostatistics (Biology 570) and Genetics (Biology 350).
%%In the Fall of 2009, I was responsible for three lab sections of Biostatistics.
%%In Spring and Fall of 2010, I was responsible for two and three discussion sections of Genetics.
%%Below are a selection of comments from my students from each appointment.
%%\begin{multicols}{2}
%\subsection*{Biology 350---Genetics---Fall 2010}
%\begin{myItemize}
%\sffamily
%\item ``Thank you so so much for reviews\ldots If you want to be in academics later, you'll be an excellent professor.''
%\item ``He was an awesome TA! Loved his review sessions---very helpful''
%\item ``The instructor presented information \underline{very} well, answered questions clearly and took the time to give thorough reviews for exams.''
%%\item ``The instructor always took time to answer questions about homework and went out of his way to help students study for exams in the class.''
%\item ``Jamie was open to questions \& explained things in a way that was easy to understand.''
%%\item ``Made hard concepts easy to understand.''
%\item ``Very helpful; often scheduled review sessions on his own time.''
%\item ``Very approachable---had lots of knowledge of material \& explained topics very well.''
%%\item ``Jamie was enthusiastic \& helpful.''
%\item ``Really great teacher. Explained things well. Better than the professors.''
%%\item ``Instructor explained well, made sure class understood material. Enjoyed the instructor, class was great!''
%%\item ``His teaching is good. He is easy to understand.''
%%\item ``He was very helpful and made himself very available for questions/reviews.''
%\end{myItemize}
%
%\subsection*{Biology 350---Genetics---Spring 2010}
%\begin{myItemize}
%\sffamily
%\item ``Jamie is the best TA!''
%\item ``Jamie is one of the best TAs I've had. Very
%helpful \& explains things very well. He never just assumed or
%expected that we knew things\ldots he took the time to explain it.''
%\item ``Jamie is one of the best TAs I've ever had. He explains the subject
%very well, answers questions and always ready to help.''
%%\item ``Very clear in his teaching. Very knowledgeable about the material.''
%\item ``Would try several ways to explain topics that were not easily
%understood.''
%%\item ``Always helpful, appreciated that he went over homework, and explained things thoroughly.''
%%\item ``He explained concepts in simple terms \& was able to clarify questions from lecture.''
%\item ``Very real \& down to earth. Available \& explains concepts well.''
%\item ``I like how Jamie is organized in discussion and also that he explains the material very well.''
%\item ``Course is well organized, interactive. Best instructor ever!''
%%\item ``I really enjoyed the class, and I appreciated the help during office hours. Also, very understanding about class conflicts.''
%\item ``He did a good job teaching a class during a difficult time---late in the day when students don't have a long attention span.''
%%\item ``Jamie is very helpful \& always responds to students needs promptly.''
%%\item ``Very calm, nice \& very helpful.''
%\item ``Jamie is a great TA. Easy to understand, knows his stuff, willing to
%take time and explain topics.''
%%\item ``He was responsive to emails \& explained things well.''
%\end{myItemize}
%
%\subsection*{Biology 570---Biostatistics---Fall 2009}
%\begin{myItemize}
%%\begin{multicols}{2}
%\sffamily
%\item ``Jamie was really nice and patient with everyone.''
%%\item ``Attentive, helpful.''
%%\item ``Taught very well.''
%\item ``Easy to talk to, approachable.''
%\item ``Very helpful in lab.''
%\item ``He was very helpful, helped me learn the material, and seemed to care
%about everyone learning and doing well.''
%%\item ``Instructor knew the material well.''
%%\item ``Willing to meet outside class.''
%\item ``Jamie was really patient and explained things very clearly.''
%%\item ``Helpful and knowledgeable TA.''
%\item ``Always willing to go out of his way to help people and make sure
%everyone is on the same page.''
%\item ``Jamie was an excellent TA. Did a great job of explaining.''
%%\item ``Lots of office hours making it easy to get help.''
%%\item ``He was very helpful and clear.''
%%\end{multicols}
%\end{myItemize}
%%\end{multicols}
